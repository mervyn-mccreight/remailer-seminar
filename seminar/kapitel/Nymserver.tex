\chapter{Nym Server}

\section{Motivation}
- TypI-II anonymisierende Remailer.
-- es gibt keine Möglichkeit zu antworten.
- Typ0 war pseudonymisierend, bot diese Möglichkeit also schon,
-- aber Betreiber besaß Zuordnung zw. Pseudonym und Identität, ist also per Definition kein Nym-Server (quelle: https://www.techopedia.com/definition/1696/nym-server)
--- einerseits dadurch Schwachstelle vorhanden
--- andererseits Betreiber dadurch angreifbar (s. typ0-remailer scientology shit)

- ziel von nym servern also: pseudonymisierung ermöglichen, ohne bekannte zuordnung zw. pseudonym und tatsächlicher identität.

\section{Umsetzung mit Hilfe von Typ-I Remailern}
- Idee: Nym-Server mit Hilfe von Typ-I-Remailer Protokoll umsetzen.
- Statt Zuordnung Pseudonym - realer Identität -> Pseudonym - Reply Block (quelle: loesing2009privacy s. 25)
-- Reply Block ermöglicht Zustellung von Nachricht über Remailer-Netzwerk an realen Empfänger
-- Zuordnung = Nym
--- beinhaltet: public-key und reply block zu seiner realen adresse und pseudonym (quelle: loesing2009privacy s. 25)
--- schickt initial reply block, sein pseudonym und public key über ein remailer netz an nym server (damit identität für nym-server geheim bleibt)
--- nun kann alice bob über pseudonym beim nym-server erreichen.
--- besser ist: mehrere nyms mit versch. reply-blöcken pro pseudonym, nym-server wählt dann zufällig einen aus, so wird nicht immer die selbe route genommen.
--- ablauf:
---- alice schickt nachricht an bobs pseudonym an nym-server
---- nym server verschl. nachricht mit bobs hinterlegtem public-key.
---- leitet verschl. nachricht mit hinterlegtem reply-block an ersten remailer in der kette im reply-block.
---- nachricht wird über kette im replay-block an bob zugestellt. (quelle: loesing2009privacy s. 26)

---- schwächen sind die selben wie verwendetes remailer-protokoll.
---- mit typ-2 nicht möglich, weil reply-block nicht erstellbar, ohne inhalt der nachricht zu kennen, wegen integritätscheck.