\chapter{Nym Server}

\section{Motivation}
In den beiden vorherigen Kapiteln wurden die Typ-I und Typ-II Remailer vorgestellt. Bei beiden handelt es sich um anonymisierende Remailer. Die Konsequenz davon ist, dass die Identität des Senders zwar verborgen bleibt, es auf diesem Weg jedoch auch unmöglich ist einem Absender einer Nachricht zu antworten. Um diese Funktionalität anbieten zu können wurden Nym-Server entwickelt.

Ein Nym-Server ist ein Pseudonym-Server, der es seinen Benutzern erlaubt für sich auf ihm ein Pseudonym zu hinterlegen. Nun sind die Benutzer mit Pseudonym über den Nym-Server von anderen Personen über deren Pseudonym erreichbar, ohne dass andere Personen die reale Identität, die sich hinter dem Pseudonym verbirgt, erfahren.

Bei dem Typ-0-Remailer war es bereits möglich einem Absender einer Nachricht auf diese zu antworten. Grund dafür ist, dass der Typ-0-Remailer auch nur pseudonymisierend war. Er ersetzte die Absenderadresse durch ein Pseudonym und besaß eine Zuordnungstabelle zwischen Pseudonymen und tatsächlichen Identitäten. Dadurch war es einem Typ-0-Remailer möglich, Nachrichten die an ein Pseudonym gesendet wurden, an die tatsächliche Adresse weiterzuleiten. Der große Nachteil dieser Technik ist jedoch, dass einem Betreiber eines Typ-0-Remailers auf diese Weise eine Zuordnung zwischen verwendeten Pseudonymen und Identität vorliegt. Einerseits stellt diese Zuordnung eine starke Schwachstelle dar, da ein Angreifer möglicherweise dazu in der Lage ist diese Zuordnung ebenfalls zu erhalten, andererseits macht sich der Betreiber selbst angreifbar. So kann es dazu kommen, dass er z.B. durch den Staat dazu gezwungen wird die Identität eines Pseudonyms preis zu geben. Per Definition handelt es sich daher bei den Typ-0-Remailern nicht um Nym-Server\footnote{vgl. \cite{nymdefinition}}.

\section{Umsetzung mit Hilfe von Typ-I Remailern}
- ziel von nym servern also: pseudonymisierung ermöglichen, ohne bekannte zuordnung zw. pseudonym und tatsächlicher identität.
- Idee: Nym-Server mit Hilfe von Typ-I-Remailer Protokoll umsetzen.
- Statt Zuordnung Pseudonym - realer Identität -> Pseudonym - Reply Block (quelle: loesing2009privacy s. 25)
-- Reply Block ermöglicht Zustellung von Nachricht über Remailer-Netzwerk an realen Empfänger
-- Zuordnung = Nym
--- beinhaltet: public-key und reply block zu seiner realen adresse und pseudonym (quelle: loesing2009privacy s. 25)
--- schickt initial reply block, sein pseudonym und public key über ein remailer netz an nym server (damit identität für nym-server geheim bleibt)
--- nun kann alice bob über pseudonym beim nym-server erreichen.
--- besser ist: mehrere nyms mit versch. reply-blöcken pro pseudonym, nym-server wählt dann zufällig einen aus, so wird nicht immer die selbe route genommen.
--- ablauf:
---- alice schickt nachricht an bobs pseudonym an nym-server
---- nym server verschl. nachricht mit bobs hinterlegtem public-key.
---- leitet verschl. nachricht mit hinterlegtem reply-block an ersten remailer in der kette im reply-block.
---- nachricht wird über kette im replay-block an bob zugestellt. (quelle: loesing2009privacy s. 26)

---- schwächen sind die selben wie verwendetes remailer-protokoll.
---- mit typ-2 nicht möglich, weil reply-block nicht erstellbar, ohne inhalt der nachricht zu kennen, wegen integritätscheck.