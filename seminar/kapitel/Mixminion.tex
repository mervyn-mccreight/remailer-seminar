\chapter{Typ-III Remailer}
- entscheidender Vorteil: Antworten auf anonyme Nachrichten (in keinem der anderen ANONYMEN Remailer Protokolle möglich (bei Typ-0 ging es, da Pseudonym).
- hat das Typ-II Remailer Protokoll als Basis (baut darauf auf)
- Weiterentwicklung dient nicht primär der Sicherheit (Mixmaster gilt schon als sehr sicher), sondern der generell der Funktionalität
- wichtige Neuerungen: 1) Antworten auf anonyme Nachrichten, 2) Verschlüsselte Kommunikation zw. den Remailern (TLS statt SMTP, verhindert Abfangen von Nachrichten) 3) Einführung von zentralem Verzeichnisserver

\section{Funktionsweise}
\subsection{SURBs}
- Bob kann eigentlich nicht Antworten, da er Identität von Alice nicht kennt.
- Dafür erstellt Alice SURB (single use reply block) und hängt ihn an Nachricht an (ist verschlüsselt).
- enthält verschlüsselte E-Mail Adresse von Alice und Pfad durch das Remailer-Netzwerk zu Alice.
- Empfänger kann SURB verwenden, um auf anonyme Nachricht zu antworten
-- Nachricht an SURB anhängen und an ersten Remailer des Remailerpfades senden.
- Verwendung nur ein mal möglich

\subsection{Nachrichten}
\subsubsection{Typisierung}
- Hinzunahme von Antworten benötigt Typisierung von Nachrichten.
- 1) normale Nachricht 2) direkte Antwort über SURB 3) anonyme Antwort (Bob bleibt auch gegenüber Alice anonym).

\subsubsection{Ununterscheidbarkeit}
- Nachrichten müssen nach außen hin Ununterscheidbar sein, damit Eve nicht über die Unterscheidung auf Rückschlüsse bzgl. des Nachrichtenverkehrs kommen kann (Nachrichtenflusstransparenz aus Mixmaster muss erhalten bleiben).

\subsubsection{Aufbau}
- Größe 32kB
- besteht aus Header und Rest (Rest enthält eigentlichen Inhalt).
- Pfad im Netzwerk wird in zwei Teile aufgeteilt -> also wird auch Header in zwei Teile aufgeteilt (primär erster Teil und sekundär zweiter Teil).
- Pfad darf maximal 32 Remailer beinhalten.
- Für jeden Remailer im Pfad enthält sowohl primärer, als auch Sekundärer Header einen (pro Remailer) Sub-Header, der Prüfsumme für den Rest des Headers enthält (um Integrität des Headers nachzuweisen).
- SubHeader beinhaltet noch: Master Secret für Erstellung von symm. Schlüssel für Verschlüsselung von Nachricht zum Übertragen (TLS) und Adresse des nächsten Remailers

- normale Nachrichten: primärer und sekundärer Header vom Absender befüllt (logisch)
- direkte Antwort: sekundärer Header mit beliebigen Daten gefüllt (?)
- anonyme Antwort: im sekundären Header ist SURB des Antworters (ursprünglicher Empfänger Bob).

\subsection{Verzeichnisserver}
- Verzeichnis auf mehrere Server verteilt
- alle VZ-Server werden ständig synchronisiert, sind also zueinander redundant
- VZ-Server verifizieren sich gegenseitig, um Einführen von manipulierten VZ-Servern zu vermeiden.
-- soll ausfallsicherheit gewährleisten
- informiert Benutzer über Status und Schlüssel der versch. Remailer im Netzwerk
- jeder Remailer im Netzwerk registriert sich anfangs beim Verzeichnisserver
- Remailer aktualisiert VZ-Server laufend über Zustand und Schlüssel

\section{Ablauf}
- Alice besorgt sich vom VZ-Server alle benötigten Informationen (Status und Schlüssel), um Pfad auszuwählen.
- Versenden und Verschlüsseln äquivalent zu Mixmaster.
- Vor Weiterleitung überprüft Remailer Integrität der Nachricht (wie Mixmaster)
- Dann baut Remailer verschlüsselte Verbindung (TLS) zum nächsten Remailer auf.
-- sym. Schlüssel dafür findet er in Header (Master Secret)
- Dann Weiterleitung

- Verhalten beim Übergang zw. ersten Teil und zweiten Teil des Pfades (swap-Operation? genauer Herausfinden...)


\section{Sicherheitsanalyse}
- theoretisch sicherster Remailer
- aber praktisch nicht nachzuweisen
- da nie in den praktischen Einsatz gekommen (blieb im Beta-Stadium, wurde nie fertig entwickelt).


