\chapter{Zusammenfassung}

In den vorausgegangenen Kapiteln wurde die Funktionsweise der Protokolle der verschiedenen Remailertypen I - III betrachtet. Die Protokolle wurden insbesondere aus sicherheitstechnischer Sicht analysiert. So wurde für jeden Remailertyp behandelt, ob er gegen gängige Angriffsarten resistent ist, und welche Sicherheitsmechanismen entwickelt wurden, um eine Immunität der verschiedenen Angriffsarten zu gewährleisten.  Da diese drei Remailertypen auf dem Prinzip der Mixnetzwerke basieren, wurden Angriffstypen gewählt, die Mixnetzwerke potentiell gefährden können.

Die verschiedenen Remailer-Protokolle unterscheiden sich im Umfang der Funktionalitäten, die das jeweilige Protokoll nativ zur Verfügung stellt. Das Typ-I und Typ-II Protokoll beschränkt sich auf die reine Funktionalität eines Remailers. Es spezifiziert, wie eine Nachricht anonymisiert übertragen werden kann. Im Typ-III Protokoll ist zusätzlich die Möglichkeit des Antwortens auf anonymisierte Nachrichten enthalten. Dies war für das Typ-I Protokoll nur über einen zusätzlichen Nym Server möglich. Des Weiteren ist ein Verzeichnis, in dem alle verwendbaren Remailer des entsprechenden Typs inklusive aller zur Benutzung notwendigen Daten aufgelistet sind, nicht Teil des Typ-I und Typ-II Protokolls. Ein Verzeichnis muss manuell gepflegt werden. Das Typ-III Protokoll umfasst auch die automatisierte Pflege eines Verzeichnisses.

\begin{table}[htbp]
	\centering
	\begin{tabular}{c}
		\textbf{Übersicht der Features}
	\end{tabular}
		
	\begin{tabular}{m{7.5cm} || c | c | c}
		\hline
		\textbf{Feature} & \textbf{Cypherpunk} & \textbf{Mixmaster} & \textbf{Mixminion} \\
		\hline
		Anonymisiertes Versenden von Nachrichten & $\surd$ & $\surd$ & $\surd$ \\
		Antworten auf anonymisierte Nachrichten & $\times$ & $\times$ & $\surd$ \\
		Verzeichnis & $\times$ & $\times$ & $\surd$ \\
	\end{tabular}

	\caption{Fähigkeiten der Remailertypen I-III}
\end{table}


Im Kontext der Ausarbeitung wurde herausgearbeitet, dass die Entwicklung der verschiedenen Remailertypen hauptsächlich fehlergetrieben war. So wurde ein neues Protokoll entwickelt, indem die Schwächen des vorherigen Protokolls analysiert wurden und versucht wurde, aufbauend auf das vorhergehende Protokoll diese Fehler zu beheben. Dementsprechend bauen die drei Remailertypen alle auf einer gemeinsamen Entwurfsbasis auf. Im Laufe der Entwicklung der Typen wurde diese Basis durch inkrementelle Erweiterungen stetig verbessert.

\clearpage

\begin{table}[htbp]
	\centering
	\begin{tabular}{c}
		\textbf{Aktive Attacken}
	\end{tabular}
		
	\begin{tabular}{m{7.5cm} || c | c | c}
		\hline
		\textbf{Angriffsart} & \textbf{Cypherpunk} & \textbf{Mixmaster} & \textbf{Mixminion} \\
		\hline
		Manipulation eines Remailers & $\surd$ & $\surd$ & $\surd$ \textbackslash ~? \\
		Wiedereinspielung von Nachrichten (Replay) & $\times$ & $\surd$ & $\surd$ \textbackslash ~? \\
		Manipulation einer Nachricht (Tagging) & $\times$ & $\surd$ & $\surd$ \textbackslash ~? \\
	\end{tabular}

	~\\[1ex]

	\begin{tabular}{c}
		\textbf{Passive Attacken}
	\end{tabular}
	
	\begin{tabular}{m{7.5cm} || c | c | c}
		\hline
		\textbf{Angriffsart} & \textbf{Cypherpunk} & \textbf{Mixmaster} & \textbf{Mixminion} \\
		\hline
		Traffic Analyse & $\times$ & $\surd$ & $\surd$ \textbackslash ~? \\
	\end{tabular}

	\caption{Mögliche Angriffe auf Typ-I bis Typ-III Remailer mit Resistenzangabe}
	\label{attacks}
\end{table}

Tabelle \ref{attacks} zeigt, dass das Cypherpunk-Remailer Protokoll Schwächen gegen diverse Angriffe aufweist. Das Mixmaster- und das Mixminion-Protokoll gelten als theoretisch sicher. Das Mixminion-Protokoll ist dem Mixmaster-Protokoll, da es sich um eine Weiterentwicklung handelt, theoretisch sicherheitstechnisch überlegen. Es wendet aktuellere, sicherheitsrelevante Forschungsergebnisse an, wie die Verwendung von verschlüsselten und verifizierten Verbindungen, wodurch das Abfangen von Paketen durch Dritte verhindert werden soll. Da die praktische Implementierung des Mixminion-Protokolls jedoch seit 2007 in der Betaphase stagniert, ist das Mixminion-Protokoll in der Praxis nicht bedenkenlos verwendbar.Daher ist aktuell das Mixmaster-Remailer Protokoll das meistverwendete Verfahren. 

Ein weiterer Betrachtungspunkt ist die Zuverlässigkeit eines Remailers. Wenn Alice eine Nachricht an Bob schickt, muss für Alice sichergestellt sein, dass Bob die Nachricht auch tatsächlich erhält. Bei allen drei Remailertypen handelt es sich um Protokolle, denen ein Verbund von Remailern zur anonymisierten Versendung einer Nachricht zugrunde liegt. Es ist gegeben, dass einem Knoten zu keiner Zeit der gesamte Pfad der Nachricht bekannt ist. Ein Knoten kennt nur den unmittelbaren Vorgänger und Nachfolger des Pfads. Aus diesem Grund ist ein Knoten im Netzwerk nicht dazu in der Lage Alice ein Fehlschlagen der Zustellung mitzuteilen. Aus diesem Grund ist es wichtig, dass die Übertragung der Nachricht durch das Netzwerk fehlerfrei zugesichert werden kann.

\begin{table}[htbp]
	\centering
	\begin{tabular}{c}
		\textbf{Zuverlässigkeitsanalyse}
	\end{tabular}
		
	\begin{tabular}{m{7.5cm} || c | c | c}
		\hline
		\textbf{Szenario} & \textbf{Cypherpunk} & \textbf{Mixmaster} & \textbf{Mixminion} \\
		\hline
		Ausfall eines Remailers im Pfad & $\times$ & $\times$ & $\surd$ \textbackslash ~? \\
		Manipulation eines Remailers im Pfad & $\times$ & $\times$ & $\surd$ \textbackslash ~? \\
	\end{tabular}

	\caption{Zuverlässigkeit des Nachrichtenzustellvorgangs der Typ-I bis Typ-III Remailer}
	\label{zuverlässigkeit}
\end{table}

Die Tabelle \ref{zuverlässigkeit} zeigt, dass sowohl im Cypherpunk, als auch im Mixmaster- Protokoll eine Zustellung einer Nachricht zum Zeitpunkt der Erstellung nicht zugesichert werden kann. In beiden Fällen ist der Grund dafür das manuell organisierte Verzeichnis der Remailer. Beim Mixminion-Protokoll wird durch die automatische und dauerhafte Synchronisierung der Remailer mit einem Verzeichnisserver der Ausfall eines Remailerknotens durch das Protokoll festgestellt und es wird verhindert, dass ein Klient diesen Remailer in seinen Pfad aufnimmt. Da das Mixminion Protokoll nie vollständig implementiert wurde, ist diese Zusage rein theoretischer Natur.