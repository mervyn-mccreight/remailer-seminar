\chapter{Typ-I Remailer}
Ca. 1994 beschloss eine Interessensgruppe mit dem Namen "Cypherpunk", die Entwicklung der Remailer voranzutreiben und entwickelten den Cypherpunk-Remailer. Ziel der Entwicklung war es, die Unsicherheiten,des Typ-0 Remailer zu beseitigen. 

Der Cypherpunk-Remailer wird als Typ-I Remailer klassifiziert. Anders als bei dem Typ-0 Remailer, der ein pseudonymisierender Remailer ist, handelt es sich bei dem Typ-I Remailer um einen anonymisierenden Remailer. Das Ziel einer Anonymisierung ist das Verändern personenbezogener Daten in der Art, dass es unmöglich ist diese Daten einer Person zuzuordnen. \footnote {vgl. § 3 Abs. 6 BDSG}
Die Anonymisierung bezieht sich in diesem Fall auf den Absender, sodass es das Ziel ist, jede Information über den Absender der Nachricht zu verstecken (horster2013datenschutz - seite 151).  


\section{Cypherpunk}

\section{Mix Netzwerke}
Die Umsetzung der Cypherpunk-Remailer basiert stark auf der Idee der Mix-Netzwerke, 

\section{Nachrichtenaustausch}

\section{Sicherheitsanalyse}
