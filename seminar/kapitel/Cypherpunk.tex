\chapter{Typ-I Remailer}
Ca. 1994 beschloss eine Interessensgruppe mit dem Namen "Cypherpunk", die Entwicklung der Remailer voranzutreiben und entwickelten den Cypherpunk-Remailer. Dieser wird als Typ-I Remailer klassifiziert. Ziel der Entwicklung war es, die Unsicherheiten des Typ-0 Remailer zu beseitigen. 
Anders als bei dem Typ-0 Remailer, der ein pseudonymisierender Remailer ist, handelt es sich bei dem Typ-I Remailer um einen anonymisierenden Remailer. Das Ziel einer Anonymisierung ist das Verändern personenbezogener Daten in der Art, dass es unmöglich ist diese Daten einer Person zuzuordnen. \footnote {vgl. § 3 Abs. 6 BDSG}
Die Anonymisierung bezieht sich in diesem Fall auf den Absender, sodass es das Ziel ist, jede Information über den Absender der Nachricht zu verstecken \footnote{vgl. S.151 \cite{horster2013datenschutz}}. Folgerichtig bietet ein anonymisierender Remailer deutlich mehr Geheimnisschutz als ein pseudonymisierender Remailer. 


\section{Cypherpunk}

\section{Mix Netzwerke}
Die technische Umsetzung des Cypherpunk-Remailers wurde sehr stark von der Idee der Mix Netzwerke von David Chaum beeinflusst. Ein Mix-Netzwerk ermöglicht anonyme Kommunikation innerhalb eines Netzwerkes. Ziel ist es, dass der Empfänger gegenüber dem Sender verborgen bleibt.
Ein Mix-Netzwerk besteht aus einer beliebig großen Menge an Mixen M. Ein Mix in einem Mix-Netzwerk ist üblicherweise ein Server, der von beliebigen Personen betrieben werden kann. Ein Mix fungiert in einem Mix-Netzwerk als Nachrichtenübermittler. Er versendet empfangene Nachrichten in der Form weiter, sodass sie nicht mehr auf angenommene Nachrichten zurückzuführen sind. Durch diese Eigenschaft wird Senderanonymität gewährleistet.

\section{Nachrichtenaustausch}
Einige Konzepte des Mix-Netzwerkes wurden aufgegriffen um den Typ-I Remailer zu entwickeln. So werden Nachrichten über mehrere Typ-I Remailer verschickt. Möchte Alice eine Nachricht N an Bob übermitteln, sucht sie sich aus einer gegebenen Menge von Cypherpunk-Remailern eine endliche Teilmenge C = (C1, C2, ..., Cn) an Remailern aus, über die die Nachricht Schritt für Schritt an Bob übertragen wird. Für diese Teilmenge definiert Alice eine Routingreihenfolge. Jeder Remailer verfügt über je einen öffentlichen Schlüssel Ec und einen privaten Schlüssel Dc.
Alice verschlüsselt nun ihre Nachricht zusammen mit den entsprechenden Routing-Informationen nacheinander mit den öffentlichen Schlüsseln Ec der selektierten Remailer, in rückwärtiger Reihenfolge der Routingordnung, beginnend mit dem letzten Remailer Cn.

(hier formelhafte oder bildhafte Erklärung der verzwiebelten Verschlüsselungsschichten. Lieber Formel als Bild).

Anschließen initialisiert sie das Versenden der Nachricht, indem sie N' an C1 schickt.

Eine Nachricht, die einen Typ-1 Remailer erreicht, enthält folgende Informationen:
\begin{itemize}
\item eine Routing-Information
\item die aktuelle Form der Nachricht
\end{itemize}




\section{Sicherheitsanalyse}
