\chapter{Typ-II Remailer}
\section{Motivation}
Das Mixmaster-Remailer Protokoll ist seit 1995 verfügbar. Die Motiviation und gleichzeitig das Ziel der Entwicklung der Typ-II Remailer war, die Schwächen der Typ-I-Remailer Generation zu beseitigen. Die wesentlichen Konzepte zur Umsetzung und Verbesserung wurden von Lance Cottrell in seiner Ausarbeitung \glqq Mixmaster and remailer attacks\grqq ~erarbeitet. Darin legt er die Schwächen der Cypherpunk-Remailer offen und analysiert, wodurch diese entstehen, und erarbeitet konkrete Vorschläge, wie die vorhandenen Sicherheitslücken zu schließen sind. Er stellt dar aus welchem Grund Cypherpunk-Remailer nicht zuverlässig verhindern, dass eine Verbindung zwischen eingehenden und ausgehenden Nachrichten an einem Knoten im Remailer-Netzwerk hergestellt werden kann \cite[S. 276]{oram2001peer}. Damit entwarf und implementierte Lance Cottrell das erste Design des Mixmaster Protokolls \cite{mixmastermanpage}.

\section{Funktionsweise}
Das Mixmaster-Remailer Protokoll basiert, analog zum Cypherpunk-Remailer Protokoll, auf einem Netzwerk von Remailern, ähnlich einem Chaum'schen Mix-Netzwerk. Das Verfahren, nach dem eine Nachricht die verschiedenen Knoten des Netzwerks traversiert, bleibt größtenteils identisch. Auch hier wird ein asymmetrisches Verschlüsselungsverfahren verwendet, auf dessen Basis eine Nachricht entsprechend der öffentlichen Schlüssel der Remailer schichtenweise verschlüsselt wird. Der Pfad einer Nachricht durch das Netzwerk muss zum Zeitpunkt der schichtenweisen Verschlüsselung vollständig bekannt sein. Weiterhin manipuliert ein Remailer nach dem Empfangen und Entschlüsseln einer Nachricht den Nachrichtenheader, um den Absender der Nachricht unkenntlich zu machen. Auf diese Weise wird jede Art von absenderbezogenen Informationen entfernt und die Nachricht anonymisiert. 

Bisher sind alle Schritte identisch mit dem Cypherpunk-Remailer Protokoll. Um die Schwächen der Typ-I Remailer zu entfernen, bedurfte es der Einführung zusätzlicher Sicherheitsmaßnahmen, die im Folgenden erläutert werden.

Da die Aufbereitung einer Nachricht für das Mixmaster-Remailer Protokoll durch die zusätzlich eingeführten Sicherheitsmechanismen sehr komplex geworden ist, existiert Client-Software für das Mixmaster-Remailer Protokoll, die das Vorbereiten einer Nachricht für einen Anwender übernehmen. Da Mixmaster-Remailer nur noch Nachrichten akzeptieren, die genau dem spezifizierten Protokoll entsprechen, ist die Verwendung einer Client-Software notwendig. Möchte Alice eine Nachricht über Mixmaster-Remailer an Bob senden, muss sie anders als bei Cypherpunk-Remailern die Vorarbeit nicht manuell durchführen. Stattdessen verwendet sie die Client-Software.

\subsection{Chunks}
Im Mixmaster-Remailer Protokoll werden Nachrichten in gleichgroße Blöcke\footnote{auch "chunks".} aufgeteilt (beispielsweise 20 kB groß). Entstehen dabei einer oder mehrere Blöcke, die nicht die gewünschte Größe haben, werden diese mit zufällig generierten Daten aufgefüllt. Anstelle der vollständigen Nachricht werden die verschiedenen gleichgroßen Blöcke einer Nachricht über das Remailer-Netzwerk verteilt. Zusammengehörende Blöcke müssen nicht denselben Pfad durch das Netzwerk nehmen. Es ist vorteilhaft, wenn die Blöcke über unterschiedliche Remailer in dem Netzwerk verteilt werden. Wichtig ist, dass der letzte Remailer, der die Nachricht an den Empfänger überträgt, für alle Blöcke einer Nachricht identisch ist. Nur dieser Remailer ist dazu in der Lage, die vollständige Nachricht wiederherzustellen, sofern er alle der Nachricht zugehörigen Blöcke empfangen hat. Anschließend leitet er die Nachricht an den Empfänger weiter.

\begin{figure}
	\begin{center}
		\def\svgwidth{0.9 \linewidth}
		\input{bilder/typ2_message_exchange.pdf_tex}
		\caption{Exemplarische Nachrichtenübertragung mit Mixmaster-Remailern}
	\end{center}
\end{figure}

Dieses Verfahren sorgt dafür, dass eine Nachricht nicht mehr anhand ihrer Größe durch das Remailer-Netzwerk verfolgt werden kann. Alle Nachrichten, die innerhalb des Remailer-Netzwerks übertragen werden, sind von gleicher Größe und Eve identisch. Eine Zuordnung ist auf diese Weise nicht mehr möglich.

\subsection{Pool}
Anders als bei den Cypherpunk-Remailern werden einkommende Nachrichten bei den Mixmaster-Remailern nicht sofort zum Zeitpunkt des Eintreffens weitergeleitet. Stattdessen speichert ein Mixmaster-Remailer seine einkommenden Nachrichten in einem Nachrichtenspeicher zwischen. Dieser Nachrichtenspeicher wird auch Nachrichtenpool genannt. Darin werden einkommende Nachrichten gesammelt. Wichtig ist, dass die Reihenfolge der Nachrichtenspeicherung hierbei keinem festen Schema folgt. Einkommende Nachrichten werden in zufälliger Reihenfolge im Nachrichtenpool abgelegt. Für jeden Remailer ist ein Größenschwellwert für den Nachrichtenpool individuell konfigurierbar. Zu dem Zeitpunkt, an dem die Größe des Nachrichtenpools diesen Schwellwert übersteigt, werden in ihm befindliche Nachrichten in zufälliger Reihenfolge an ihren entsprechenden Empfänger weitergeleitet. 

Vorstellbar ist, dass ein Remailer nie genügend Nachrichten empfängt um seinen Nachrichtenpool ausreichend zu füllen. Damit die bis dahin im Nachrichtenpool gesammelten Nachrichten nicht blockiert werden, wird nach Ablauf eines individuell festlegbaren Zeitintervalls der Nachrichtenpool um zufällig generierte Pseudonachrichten erweitert, sodass der Größenschwellwert überschritten wird. Anschließend werden alle in ihm befindlichen Nachrichten, inklusive der Attrappen, weitergeleitet. 

Auf diese Weise kann Eve eine Nachricht nicht mehr über eine zeitliche Zuordnung verfolgen. Eine Verbindung zwischen Empfangszeitpunkt und Absendezeitpunkt einer Nachricht an einem Remailer kann nicht hergestellt werden. Durch die gleichartigen Nachrichten\footnote{gleichartig bezogen auf ihre Größe.} ist es einem Angreifer zusätzlich nicht möglich, überhaupt eine Verbindung zwischen einer einkommenden und einer ausgehenden Nachricht herzustellen. 


\subsection{Signatur}
Weiterhin wurde beim Mixmaster-Remailer die Überprüfung der Integrität einer im Remailer-Netzwerk verschickten Nachricht als zusätzlicher Sicherheitsmechanismus eingeführt. Dazu wird in einem verschlüsselten Header der Nachricht eine Signatur übertragen. 
Mithilfe der Signatur prüft ein Remailer den Inhalt der Nachricht auf Manipulation oder vollständige Fremdeinführung.
In der Konsequenz ist Eve außerstande, manipulierte Nachrichten in das Remailernetzwerk einzuspielen, um das Verhalten auf diese Nachrichten zu analysieren.

\subsection{Identifikation}
Zusätzlich zur Signatur enthält eine Nachricht in einem weiteren verschlüsselten Header eine eindeutige Identifikation, üblicherweise eine Nummer. Über diese ID kann ein Remailer eine Nachricht eindeutig erkennen und ermitteln, ob er dieselbe Nachricht mehrfach empfängt. Hat ein Mixmaster-Remailer eine Nachricht bereits empfangen und sie in seinem Nachrichtenpool abgelegt, oder sogar weitergeleitet, wird er jede weitere Nachricht mit der selben Identifikation ignorieren. Durch diesen Sicherheitsmechanismus hat Eve nicht mehr die Möglichkeit, die bei Cypherpunk-Remailern denkbaren, Replay-Angriffe erfolgreich zu fahren. Fängt Eve eine Nachricht von Alice an einen Remailer ab, und versucht durch mehrmaliges Absenden der Nachricht an einen Remailer das Verhalten zu analysieren, schlägt der Angriff fehl, da mehrfach gesendete Nachrichten ignoriert werden.

\section{Sicherheitsanalyse}
Das Mixmaster-Remailer Protokoll gilt heute als sicher. Dies spiegelt sich auch in der aktuellen Benutzung von Remailern wieder. Mixmaster-Remailer sind das aktuell am meisten verwendetet Remailer-Protokoll.

Aus theoretischer Sicht sind auch Mixmaster-Remailer nicht in der Lage vollständige Anonymität zu garantieren. Eve ist es mittels einer Flooding-Attacke möglich, eine Nachricht vom Sender durch das Remailer-Netzwerk bis zum Empfänger zu verfolgen. Dabei fängt Eve die Nachricht von Alice an den ersten Remailer-Netzwerkknoten ab und hält diese zurück. Anschließend überflutet\footnote{daher der Name "Flooding-Attacke".} sie das Remailer-Netzwerk solange mit eigenen Nachrichten, bis die Nachrichtenspeicher der einzelnen Remailer aussschließlich mit Eves Nachrichten gefüllt sind. Nun ist jeder Nachrichtenverkehr im Remailer-Netzwerk auf Nachrichten von Eve zurückzuführen. Eve ist in der Lage, jede dieser Nachrichten als ihre zu identifizieren, da sie den Empfänger ihrer eigenen Nachricht kennt. Ist dieser Status erreicht, versendet Eve die von Alice abgefangene und zurückgehaltene Nachricht. Da Eve alle anderen Nachrichten im Remailer-Netzwerk kennt, kann sie die Nachricht von Alice über das gesamte Netzwerk bis zum Empfänger Bob zu verfolgen. Eve muss nur auf eine Nachricht warten, die nicht an den Empfänger ihrer Nachricht versendet wird. Der Empfänger dieser Nachricht muss der Empfänger der Nachricht von Alice, also Bob, sein.

Ein solcher Angriff ist in der Praxis nicht umsetzbar, da davon auszugehen ist, dass Mixmaster-Remailer zu jeder Zeit auch von fremden Personen verwendet werden. Zusätzlich müsste Eve nicht nur alle Remailer des Netzwerks kennen, sondern auch deren individuellen Nachrichtenpoolgrößen. All diese Fakten zu erlangen ist für Eve in der Praxis kaum realisierbar. Dadurch kann Eve zu keinem Zeitpunkt den Zustand herbeiführen, in dem nur ihre Nachrichten im Remailernetwerk vorhanden sind.